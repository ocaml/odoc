\section{Module \ocamlinlinecode{Functor}}\label{Functor}%
\label{Functor-module-type-S}\ocamlcodefragment{\ocamltag{keyword}{module} \ocamltag{keyword}{type} \hyperref[Functor-module-type-S]{\ocamlinlinecode{S}}}\ocamlcodefragment{ = \ocamltag{keyword}{sig}}\begin{ocamlindent}\label{Functor-module-type-S-type-t}\ocamlcodefragment{\ocamltag{keyword}{type} t}\\
\end{ocamlindent}%
\ocamlcodefragment{\ocamltag{keyword}{end}}\\
\label{Functor-module-type-S1}\ocamlcodefragment{\ocamltag{keyword}{module} \ocamltag{keyword}{type} \hyperref[Functor-module-type-S1]{\ocamlinlinecode{S1}}}\ocamlcodefragment{ = \ocamltag{keyword}{sig}}\begin{ocamlindent}\subsubsection{Parameters\label{Functor-module-type-S1-parameters}}%
\label{Functor-module-type-S1-argument-1-_}\ocamlcodefragment{\ocamltag{keyword}{module} \hyperref[Functor-module-type-S1-argument-1-_]{\ocamlinlinecode{\_\allowbreak{}}}}\ocamlcodefragment{ : \ocamltag{keyword}{sig}}\begin{ocamlindent}\label{Functor-module-type-S1-argument-1-_-type-t}\ocamlcodefragment{\ocamltag{keyword}{type} t}\\
\end{ocamlindent}%
\ocamlcodefragment{\ocamltag{keyword}{end}}\\
\subsubsection{Signature\label{Functor-module-type-S1-signature}}%
\label{Functor-module-type-S1-type-t}\ocamlcodefragment{\ocamltag{keyword}{type} t}\\
\end{ocamlindent}%
\ocamlcodefragment{\ocamltag{keyword}{end}}\\
\label{Functor-module-F1}\ocamlcodefragment{\ocamltag{keyword}{module} \hyperref[Functor-F1]{\ocamlinlinecode{F1}}}\ocamlcodefragment{ (\hyperref[Functor-F1-argument-1-Arg]{\ocamlinlinecode{Arg}} : \hyperref[Functor-module-type-S]{\ocamlinlinecode{S}}) : \hyperref[Functor-module-type-S]{\ocamlinlinecode{S}}}\\
\label{Functor-module-F2}\ocamlcodefragment{\ocamltag{keyword}{module} \hyperref[Functor-F2]{\ocamlinlinecode{F2}}}\ocamlcodefragment{ (\hyperref[Functor-F2-argument-1-Arg]{\ocamlinlinecode{Arg}} : \hyperref[Functor-module-type-S]{\ocamlinlinecode{S}}) : \hyperref[Functor-module-type-S]{\ocamlinlinecode{S}} \ocamltag{keyword}{with} \ocamltag{keyword}{type} \hyperref[Functor-module-type-S-type-t]{\ocamlinlinecode{t}} = \hyperref[Functor-F2-argument-1-Arg-type-t]{\ocamlinlinecode{Arg.\allowbreak{}t}}}\\
\label{Functor-module-F3}\ocamlcodefragment{\ocamltag{keyword}{module} \hyperref[Functor-F3]{\ocamlinlinecode{F3}}}\ocamlcodefragment{ (\hyperref[Functor-F3-argument-1-Arg]{\ocamlinlinecode{Arg}} : \hyperref[Functor-module-type-S]{\ocamlinlinecode{S}}) : \ocamltag{keyword}{sig} .\allowbreak{}.\allowbreak{}.\allowbreak{} \ocamltag{keyword}{end}}\\
\label{Functor-module-F4}\ocamlcodefragment{\ocamltag{keyword}{module} \hyperref[Functor-F4]{\ocamlinlinecode{F4}}}\ocamlcodefragment{ (\hyperref[Functor-F4-argument-1-Arg]{\ocamlinlinecode{Arg}} : \hyperref[Functor-module-type-S]{\ocamlinlinecode{S}}) : \hyperref[Functor-module-type-S]{\ocamlinlinecode{S}}}\\
\label{Functor-module-F5}\ocamlcodefragment{\ocamltag{keyword}{module} \hyperref[Functor-F5]{\ocamlinlinecode{F5}}}\ocamlcodefragment{ () : \hyperref[Functor-module-type-S]{\ocamlinlinecode{S}}}\\

\section{Module \ocamlinlinecode{Functor.\allowbreak{}F1}}\label{Functor-F1}%
\subsection{Parameters\label{Functor-F1-parameters}}%
\label{Functor-F1-argument-1-Arg}\ocamlcodefragment{\ocamltag{keyword}{module} \hyperref[Functor-F1-argument-1-Arg]{\ocamlinlinecode{Arg}}}\ocamlcodefragment{ : \ocamltag{keyword}{sig}}\begin{ocamlindent}\label{Functor-F1-argument-1-Arg-type-t}\ocamlcodefragment{\ocamltag{keyword}{type} t}\\
\end{ocamlindent}%
\ocamlcodefragment{\ocamltag{keyword}{end}}\\
\subsection{Signature\label{Functor-F1-signature}}%
\label{Functor-F1-type-t}\ocamlcodefragment{\ocamltag{keyword}{type} t}\\



\section{Module \ocamlinlinecode{Functor.\allowbreak{}F2}}\label{Functor-F2}%
\subsection{Parameters\label{Functor-F2--parameters}}%
\label{Functor-F2--argument-1-Arg}\ocamlcodefragment{\ocamltag{keyword}{module} \hyperref[Functor-F2-argument-1-Arg]{\ocamlinlinecode{Arg}}}\label{Functor-F2-argument-1-Arg}\ocamlcodefragment{ : \ocamltag{keyword}{sig}}\begin{ocamlindent}\label{Functor-F2-argument-1-Arg--type-t}\ocamlcodefragment{\ocamltag{keyword}{type} t}\\
\end{ocamlindent}%
\ocamlcodefragment{\ocamltag{keyword}{end}}\\
\subsection{Signature\label{Functor-F2--signature}}%
\label{Functor-F2--type-t}\ocamlcodefragment{\ocamltag{keyword}{type} t = \hyperref[Functor-F2-argument-1-Arg--type-t]{\ocamlinlinecode{Arg.\allowbreak{}t}}}\\



\section{Module \ocamlinlinecode{Functor.\allowbreak{}F3}}\label{Functor-F3}%
\subsection{Parameters\label{Functor-F3-parameters}}%
\label{Functor-F3-argument-1-Arg}\ocamlcodefragment{\ocamltag{keyword}{module} \hyperref[Functor-F3-argument-1-Arg]{\ocamlinlinecode{Arg}}}\ocamlcodefragment{ : \ocamltag{keyword}{sig}}\begin{ocamlindent}\label{Functor-F3-argument-1-Arg-type-t}\ocamlcodefragment{\ocamltag{keyword}{type} t}\\
\end{ocamlindent}%
\ocamlcodefragment{\ocamltag{keyword}{end}}\\
\subsection{Signature\label{Functor-F3-signature}}%
\label{Functor-F3-type-t}\ocamlcodefragment{\ocamltag{keyword}{type} t = \hyperref[Functor-F3-argument-1-Arg-type-t]{\ocamlinlinecode{Arg.\allowbreak{}t}}}\\



\section{Module \ocamlinlinecode{Functor.\allowbreak{}F4}}\label{Functor-F4}%
\subsection{Parameters\label{Functor-F4-parameters}}%
\label{Functor-F4-argument-1-Arg}\ocamlcodefragment{\ocamltag{keyword}{module} \hyperref[Functor-F4-argument-1-Arg]{\ocamlinlinecode{Arg}}}\ocamlcodefragment{ : \ocamltag{keyword}{sig}}\begin{ocamlindent}\label{Functor-F4-argument-1-Arg-type-t}\ocamlcodefragment{\ocamltag{keyword}{type} t}\\
\end{ocamlindent}%
\ocamlcodefragment{\ocamltag{keyword}{end}}\\
\subsection{Signature\label{Functor-F4-signature}}%
\label{Functor-F4-type-t}\ocamlcodefragment{\ocamltag{keyword}{type} t}\\



\section{Module \ocamlinlinecode{Functor.\allowbreak{}F5}}\label{Functor-F5}%
\subsection{Parameters\label{Functor-F5-parameters}}%
\subsection{Signature\label{Functor-F5-signature}}%
\label{Functor-F5-type-t}\ocamlcodefragment{\ocamltag{keyword}{type} t}\\



